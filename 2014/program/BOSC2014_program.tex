\documentclass[12pt,oneside]{article}
\usepackage[margin=2cm]{geometry}
\usepackage{multirow}
\usepackage{graphicx}
\usepackage[colorlinks=true,citecolor=black,urlcolor=blue]{hyperref}

%\newcommand{\embed}[5]{\begin{center}\fbox{\includegraphics[trim=#1 #2 #3 #4,clip,width=\textwidth]{#5}}\end{center}\clearpage}
\newcommand{\embed}[5]{\begin{center}\includegraphics[trim=#1 #2 #3 #4,clip,width=\textwidth]{#5}\end{center}\clearpage}

\usepackage{fancyhdr}
\setlength\headheight{2cm}
\fancyhead{}
\fancyfoot{}
\renewcommand{\headrulewidth}{1pt}
\renewcommand{\footrulewidth}{0pt}
\lhead{Bioinformatics Open Source Conference (BOSC) 2014\\Boston}
\rhead{\includegraphics[height=1cm]{../../logos/100px-Pear.png}}

\newcommand{\session}[3]{\clearpage\lhead{BOSC 2014 -- \textit{#1}\\#2\\#3}\newpage}

\title{%Hack to get the logo on the PDF front page:
$15^{th}$ Annual \\
Bioinformatics Open Source Conference\\
BOSC 2014\\
~\\
\includegraphics[width=4cm]{../../logos/100px-Pear.png}\\
%Hack to get some white space using a blank line:
~\\
\url{http://www.open-bio.org/wiki/BOSC_2014}
}

\date{11 \& 12 July 2014, Boston, MA, USA}

\begin{document}
\pagestyle{fancy}

\maketitle

\thispagestyle{empty}

\session{Introduction}{Chairs: Nomi Harris \& Peter Cock}{}

\section*{Welcome to BOSC 2014!}
The Bioinformatics Open Source Conference, established in 2000,
is held every year as a Special Interest Group (SIG)
meeting in conjunction with the Intelligent Systems for Molecular Biology
(ISMB) Conference. BOSC is organised by the Open Bioinformatics Foundation
(O$|$B$|$F), a non-profit group dedicated to promoting the practice and
philosophy of Open Source software development and Open Science within
the biological research community.

\section*{Talks and Posters}

BOSC includes two full days of talks, posters, and Birds-of-a-Feather
interest groups (BOFs). Session topics this year include Software
Interoperability, Visualization, Genome-Scale Data and Beyond, and
Open Science and Reproducible Research, as well as the usual session
on Bioinformatics Open Source Project Updates. Our panel topic this
year is ``Reproducibility: Rewards and Challenges''. This year's
keynote speakers will be Philip Bourne (NIH) and C. Titus Brown
(Michigan State University).

There are poster sessions both days that start during the lunch hour.
We have space for several last-minute posters, in addition to those
listed in the program. Please contact us at bosc@open-bio.org if you'd
like to present a last-minute poster.

\section*{Sponsors}

We thank Eagle Genomics for sponsoring the BOSC Student Travel Awards
again this year, and welcome the open access journal
\href{http://www.gigasciencejournal.com/}{GigaScience}, and
\href{https://curoverse.com/}{Curoverse}
(the team behind the open source platform \href{http://arvados.org/}{Arvados})
as new sponsors for BOSC 2014, along with
\href{https://developers.google.com/open-source/}{Google}
for their generous support for video-recording the talks at BOSC 2014.

\begin{center}
\begin{tabular}{cccccc}
\multirow{2}{*}{\raisebox{\dimexpr-\height+\ht\strutbox-5mm}{\includegraphics[width=3.5cm]{../../logos/120px-Eagle_logo.jpg}}} &
\quad\quad &
\raisebox{\dimexpr-\height+\ht\strutbox}{\includegraphics[width=6cm]{../../logos/175px-Gigascience-07.png}} &
\quad\quad &
\raisebox{\dimexpr-\height+\ht\strutbox-10mm}{\includegraphics[width=4cm]{../../logos/Curoverse.png}} \\
&& \raisebox{\dimexpr-\height+\ht\strutbox-5mm}{\includegraphics[width=5.5cm]{../../logos/205px-Google-logo11w.png}}
&& \raisebox{\dimexpr-\height+\ht\strutbox-5mm}{\includegraphics[width=4cm]{../../logos/Arvados.png}} &
\end{tabular}
\end{center}


\session{Organizers}{Chairs: Nomi Harris \& Peter Cock}{}
\section*{Organizers}

BOSC is a community effort -- we thank all those who made it possible,
including the organizing committee, the program committee, the session
chairs, and the ISMB SIG chair, Steven Leard. If you are interested
in helping to organize BOSC 2015, please email bosc@open-bio.org.

\subsection*{BOSC 2014 Organizing Committee}
Nomi Harris and Peter Cock (Co-Chairs),
Raoul Jean Pierre Bonnal, Brad Chapman, Robert Davey,
Christopher Fields, Hans-Rudolf Hotz, Hilmar Lapp.

\subsection*{BOSC 2014 Program Committee}
Tiago Ant{\~a}o, Kazuharu Arakawa, Raoul Bonnal, Timothy Booth,
Brad Chapman, Peter Cock, Kam Dahlquist, Robert Davey, Thomas Down,
Chris Fields, Bj{\"o}rn Gr{\"u}ning, Nomi Harris, Michael Heuer,
Hans-Rudolf Hotz, Amye Kenall, Juli Klemm, Hilmar Lapp,
Heikki Lehv{\"a}slaiho, Scott Markel, Herv{\'e} M{\'e}nager,
Fiona Nielsen, Lorena Pantano, Michael Reich, Francesco Strozzi,
Eric Talevich, Ronald Taylor, Ben Temperton.

\session{Group Dinner}{Day One, 11 July 2014, 19:00 --}{The Asgard Irish Pub and Restaurant}
\section*{Optional BOSC Dinner}

We invite you to join BOSC organizers and attendees at a pay-your-own-way
dinner the first evening of BOSC (Friday, July 11, at 7pm) at The Asgard
Irish Pub and Restaurant, 350 Massachusetts Avenue, Cambridge, MA. It is
about 1.2 miles north of the Hynes convention center on Mass Ave. The \#1
bus (towards Harvard Square) goes right there.

If you want to join us for dinner, RSVP at \url{http://bit.ly/BOSC2014-dinner}
before Friday at 3pm. The restaurant has space for 30 BOSC guests;
\textbf{only those who RSVP will be admitted}.

%TODO - insert map?

\session{CodeFest}{Pre-BOSC, 9 \& 10 July 2014}{Organisers: Brad Chapman \& Michael Heuer}

\section*{CodeFest}

\noindent
As in recent years, for the two days before BOSC we are holding the
BOSC CodeFest, an informal ``Coding Festival'' or mini-hackathon:
\url{http://www.open-bio.org/wiki/Codefest_2014}

\begin{center}
\includegraphics[width=4cm]{../../logos/Hack-reduce-logo.png}
\end{center}

\noindent
The BOSC CodeFest 2014 is being hosted by \href{http://www.hackreduce.org/}{hack/reduce},
a wonderful hacker space in Cambridge.

\begin{center}
\includegraphics[width=5cm]{../../logos/150px-Aws-logo.jpeg}
\end{center}

\noindent
Thanks to \href{http://aws.amazon.com/}{Amazon Web Services} all participants will receive a $\$100$
AWS credit to support work at the hackathon.

\begin{center}
\begin{tabular}{cc}
\multirow{2}{*}{\raisebox{\dimexpr-\height+\ht\strutbox}{\includegraphics[height=3cm]{../../logos/Harbinger-Partners.png}}} &
\raisebox{\dimexpr-\height+\ht\strutbox}{\includegraphics[width=4cm]{../../logos/Curoverse.png}} \\
& \raisebox{\dimexpr-\height+\ht\strutbox}{\includegraphics[width=4cm]{../../logos/Arvados.png}}
\end{tabular}
\end{center}

\noindent
The CodeFest has also been sponsored by \href{http://harbinger-partners.com/}{Harbinger Partners, Inc.}
and \href{https://curoverse.com/}{Curoverse} (the team behind the open source platform
\href{http://arvados.org/}{Arvados}).

\session{Keynote Speakers}{Days One \& Two, 9 \& 10 July 2014}{}
\section*{Keynote Speakers}

\subsection*{Philip Bourne (Day Two)}
\noindent
Dr. Bourne will speak about ``\textbf{Biomedical Research as an Open Digital Enterprise}'':
\begin{quote}
\small
\textit{The biomedical research lifecycle is fast becoming completely digital
and increasingly open to the point that publishing could simply become changing
the access control on given research objects comprising ideas, hypotheses, data,
software, results, conclusions, reviews, grants and so on. This offers immense
opportunities for software developers to enable the enterprise. I will describe
a vision for the digital enterprise and what the NIH and others are doing to
support the notion with the intent to accelerate scientific discovery.}
\end{quote}
Philip E. Bourne, PhD, is the Associate Director for Data Science at the NIH
and formerly Associate Vice Chancellor for Innovation and Industry Alliances,
and Professor of Pharmacology at the University of California San Diego.
Bourne's work at the NIH focuses on accelerating the rate of knowledge
discovery from the ever-increasing amounts of biomedical data at all scales --
from genomes to populations.
Bourne's laboratory focuses on relevant biological and educational outcomes
derived from computation and scholarly communication. This implies algorithms,
text mining, machine learning, metalanguages, biological databases, and
visualization applied to problems in systems pharmacology, evolution, cell
signaling, apoptosis, immunology and scientific dissemination. He has published
over 300 papers and 5 books. Bourne is the co-founder and founding Editor-in-Chief
of the open access journal PLOS Computational Biology. He is a Past President of
the International Society for Computational Biology, an elected fellow of the
American Association for the Advancement of Science (AAAS), the International
Society for Computational Biology (ISCB) and the American College of Medical
Informatics.
Bourne is committed to professional development through the Ten Simple Rules
series of articles and a variety of lectures and video presentations. Awards
include: the Jim Gray eScience Award (2010), the Benjamin Franklin Award (2009),
the Flinders University Convocation Medal for Outstanding Achievement (2004),
the Sun Microsystems Convergence Award (2002) and the CONNECT Award for new
inventions (1996 \& 97).

\subsection*{C.~Titus Brown (Day One)}
\noindent
Dr. Brown's topic is ``\textbf{A History of Bioinformatics (in the year 2039)}''.
\begin{quote}
\small
\textit{In 2039, I expect to look back at the last 25 years of biology and see
both wonderful surprises and missed opportunities. In this talk, I will attempt
to predict both some surprises and some of the opportunities I worry that we
will have missed over the next 25 years.}
\end{quote}
C.~Titus Brown is an assistant professor in the Department of Computer Science
and Engineering and the Department of Microbiology and Molecular Genetics. He
earned his PhD ('06) in developmental molecular biology from the California
Institute of Technology. Brown is director of the laboratory for Genomics,
Evolution, and Development (GED) at Michigan State University. He is a member
of the Python Software Foundation and an active contributor to the open source
software community. His research interests include computational biology,
bioinformatics, open source software development, and software engineering.

\session{Schedule}{Day One, 9 July 2014}{}
\section*{Schedule - Day One}

Please refer to \url{http://www.open-bio.org/wiki/BOSC_2014_Schedule} for the
latest schedule, including any changes since this document was prepared.
Abstracts follow later in this document.

\begin{center}
\begin{tabular}{|p{2.4cm}|p{10cm}|p{4.2cm}|}
\hline
\textbf{7:30-9:00} & \textbf{Registration} &\\
9:00-9:15 & Introduction and welcome &\\
9:15-10:15 & \textbf{Keynote}: A History of Bioinformatics (in the year 2039) & C. Titus Brown\\
\hline
\textbf{10:15-10:45} & \textbf{Coffee Break} &\\
\hline
\textit{10:45-12:30} & \textbf{\textit{Session: Genome-scale Data and Beyond}} & Chair: Chris Fields\\
10:45-11:03 & ADAM: Fast, Scalable Genomic Analysis & Frank Austin Nothaft\\
11:03-11:21 & A framework for benchmarking RNA-seq pipelines & Rory Kirchner\\
11:21-11:39 & New Frontiers of Genome Assembly with SPAdes 3.1 & Andrey Prjibelski\\
11:39-11:57 & SigSeeker: An Ensemble for Analysis of Epigenetic Data & Jens Lichtenberg\\
11:57-12:15 & Galaxy as an Extensible Job Execution Platform & John Chilton\\
12:15-12:30 & Open Bioinformatics Foundation (OBF) Update & Hilmar Lapp\\
\hline
\textbf{12:30-13:30} & \textbf{Lunch} &\\
\textbf{13:00-14:00} & \textbf{Poster Session and Birds of a Feather (BOFs)} &  (overlapping w/ lunch)\\
\hline
\textit{14:00-15:30} & \textbf{\textit{Session: Visualization}} & Chair: Rob Davey\\
14:00-14:18 & WormGUIDES: an Interactive Informatic Developmental Atlas at Subcellular Resolution & Anthony Santella\\
14:18-14:36 & BioJS: an open source standard for biological visualisation & Manuel Corpas\\
14:36-14:54 & Biodalliance: a fast, extensible genome browser & Thomas Down\\
14:54-15:12 & TGAC Browser: visualisation solutions for big data in the genomic era & Anil S. Thanki\\
15:12-15:30 & Explore, analyze, and share genomic data using Integrated Genome Browser & Ann Loraine\\
\hline
\textbf{15:30-16:00} & \textbf{Coffee Break} &\\
\hline
\end{tabular}
\end{center}

\section*{Schedule - Day One (continued)}

Please refer to \url{http://www.open-bio.org/wiki/BOSC_2014_Schedule} for the
latest schedule, including any changes since this document was prepared.
Abstracts follow later in this document.

\begin{center}
\begin{tabular}{|p{2.4cm}|p{10cm}|p{4.2cm}|}
\hline
\textit{16:00-17:00} & \textbf{\textit{Session: Project Updates}} & Chair: Peter Cock\\
16:00-16:12 & BioMart 0.9 -- introducing tools for data analysis and visualisation & Luca Pandini\\
16:12-16:24 & Biocaml: The OCaml Bioinformatics Library & Ashish Agarwal\\
16:24-16:36 & BioRuby and distributed development & Pjotr Prins\\
16:36-16:48 & Biopython Project Update & Wibowo Arindrarto\\
16:48-17:00 & Shared bioinformatics database within Unipro UGENE & Ivan Protsyuk\\
\hline
\textit{17:00-17:30} & \textbf{\textit{Session: Lightning Talks}} & Chair: Peter Cock\\
17:00-17:05 & Fostering the next generation of data-driven open science with R & Karthik Ram\\
17:07-17:12 & Tripal: an open source toolkit for building genomic and genetic data websites and databases & Margaret Staton\\
17:14-17:19 & PLUTo: Phyloinformatic Literature Unlocking Tools & Ross Mounce\\
17:21-17:26 & A publication model that aligns with the key Open Source Software principles & Michael L. Markle\\
17:27-17:30 & Announcements & \\
\hline
17:30-18:30 & \textbf{Birds of a Feather (BOFs)} & \\
19:00-- & \textit{Pay-your-own-way BOSC dinner} & \\
\hline
\end{tabular}
\end{center}

\session{Schedule}{Day Two, 10 July 2014}{}
\section*{Schedule - Day Two}

Please refer to \url{http://www.open-bio.org/wiki/BOSC_2014_Schedule} for the
latest schedule, including any changes since this document was prepared.
Abstracts follow later in this document.

\begin{center}
\begin{tabular}{|p{2.4cm}|p{10cm}|p{4.2cm}|}
\hline
8:45-8:50 & Announcements & \\
8:50-9:00 & Codefest 2014 Report & Brad Chapman\\
9:00-9:15 & Open Bioinformatics Foundation (OBF) update & Hilmar Lapp\\
9:15-10:15 & Keynote: Biomedical Research as an Open Digital Enterprise & Philip Bourne\\
\hline
\textbf{10:15-10:45} & \textbf{Coffee Break} & \\
\hline
\textit{10:45-12:30} & \textbf{\textit{Session: Software Interoperability}} & Chair: Raoul Bonnal\\
10:45-11:03 & Pathview: an R/Bioconductor package for pathway-based data integration and visualization & Weijun Luo\\
11:03-11:21 & Use of semantically annotated resources in the Mobyle2 Web Framework & Herv{'e} M{'e}nager\\
11:21-11:39 & Towards ubiquitous OWL computing: Simplifying programmatic authoring of and querying with OWL axioms & Hilmar Lapp\\
11:39-11:57 & Integrating Taverna Player into Scratchpads & Robert Haines\\
11:57-12:15 & Small tools for Bioinformatics & Pjotr Prins\\
\hline
\textbf{12:30-13:30} & \textbf{Lunch} &\\
\textbf{13:00-14:00} & \textbf{Poster Session and Birds of a Feather (BOFs)} &  (overlapping w/ lunch)\\
\hline
\textit{14:00-15:30} & \textbf{\textit{Session: Open Science and Reproducible Research}} & Chair: Hilmar Lapp\\
14:00-14:18 & SEEK for Science: A Data Management Platform which Supports Open and Reproducible Science & Carole Goble\\
14:18-14:36 & Arvados: Achieving Computational Reproducibility and Data Provenance in Large-Scale Genomic Analyses & Brett Smith\\
14:36-14:54 & Enhancing the Galaxy Experience through Community Involvement & Daniel Blankenberg\\
14:54-15:12 & Supporting dynamic community developed biological pipelines & Brad Chapman\\
15:12-15:30 & Open as a strategy for durability, reproducibility and scalability & Jonathan Rees\\
\hline
\textbf{15:30-16:00} & \textbf{Coffee Break} &\\
\hline
16:00-17:00 & \textbf{Panel}: Reproducibility: Rewards and Challenges & Chair: Brad Chapman\\
17:00-17:10 & Presentation of Student Travel Awards \& Concluding Remarks & \\
17:10-18:00 & \textbf{Birds of a Feather (BOFs)} & \\
\hline
\end{tabular}
\end{center}

\session{Talk and Poster Abstracts}{Days One \& Two, 9 \& 10 July 2014}{}
\section*{Talk and Poster Abstracts}

\subsection*{Main Talks}
Talk abstracts are included in this program in the order in which they will
be presented at the conference. These have been divided into five sessions:
\begin{itemize}
\item Genome-scale Data and Beyond
\item Visualization
\item Bioinformatics Open Source Project Updates
\item Software Interoperability
\item Open Science and Reproducible Research
\end{itemize}

This year we have allocated all the talks an equal slot, 15 minutes plus 3
minutes for questions (making 18 minutes), except for the project update
session where this is reduced to 10 minutes plus 2 minutes for questions
(making 12 minutes in all).

\subsection*{Lightning}
Lightning talks are 5 minutes only (no questions), and are intended for a
brief introduction with the expectation that interested people will talk
in the evening or next day.

There should also a few spaces available for last-minute Lightning talks.
If you would like to present one (which must meet the BOSC criteria of freely
available source and recognized open source license), contact bosc@open-bio.org.

\subsection*{Posters}

In addition to the talks, there will also be posters. Some, but not all, of the
talks will also be presented as posters.

Authors should put up their posters in their assigned poster spot before the
first poster session (which starts at 12:30 on the first day). After that time,
any unused poster slots will be made available for last-minute posters.

The ISMB staff specify that posters should not exceed the following dimensions:
46 inches (1.17m) wide by 45 inches (1.14m) high.

There should also a few spaces available for last-minute posters. If you would
like to present one, please email your abstract (which must meet the BOSC
criteria of freely available source and recognized open source license) to
bosc@open-bio.org.

\session{Genome-scale Data and Beyond}{Day One, 11 July 2014, 10:45 -- 12:30}{Session chair: Chris Fields}
\embed{2.5cm}{10cm}{2.5cm}{2.5cm}{GenomeScale-16-ADAM.pdf}
\embed{2.5cm}{10cm}{2.5cm}{2.5cm}{GenomeScale-32-Framework-RNASeq.pdf}
\embed{2.5cm}{4cm}{2cm}{2.5cm}{GenomeScale-27-SPADES.pdf}
\embed{2cm}{2.5cm}{1cm}{2.5cm}{GenomeScale-22-SigSeeker.pdf}
\embed{2.5cm}{3cm}{1.5cm}{2.5cm}{GenomeScale-21-Galaxy-extending.pdf}

\session{Visualization}{Day One, 11 July 2014, 14:00 -- 15:30}{Session chair: Rob Davey}

\embed{2.5cm}{2.75cm}{1cm}{2.5cm}{Visualization-39-WormGUIDES.pdf}
\embed{3cm}{10cm}{3cm}{2.5cm}{Visualization-05-BioJS.pdf}
\embed{2.5cm}{11cm}{2.5cm}{2cm}{Visualization-07-Biodalliance.pdf}
\embed{2cm}{2.5cm}{0cm}{2.5cm}{Visualization-17-TGAC-browser.pdf}
\embed{2.5cm}{10cm}{2.5cm}{2.5cm}{Visualization-38-IGB.pdf}

\session{Bioinformatics Open Source Project Updates}{Day One, 11 July 2014, 16:00 -- 17:00}{Session chair: Peter Cock}

\embed{2.5cm}{8cm}{2.5cm}{2.5cm}{Updates-20-BioMart.pdf}
\embed{2.5cm}{3cm}{1cm}{2.5cm}{Updates-25-Biocaml.pdf}
\embed{2cm}{2.5cm}{2cm}{3cm}{Updates-41-BioRuby.pdf}
\embed{2cm}{1.5cm}{0cm}{1cm}{Updates-19-Biopython.pdf}
\embed{2.25cm}{3cm}{1.5cm}{2.5cm}{Updates-02-UGENE.pdf}

\session{Five Minute Lightning Talks}{Day One, 11 July 2014, 17:00 -- 17:30}{Session chair: Peter Cock}

\embed{2.5cm}{14cm}{2.5cm}{2.5cm}{Lightning-43-ForsteringR.pdf}
\embed{2.5cm}{3.5cm}{1.5cm}{2.5cm}{Lightning-46-Tripal.pdf}
\embed{1cm}{2cm}{0cm}{2.5cm}{Lightning-10-PLUTo.pdf}
\embed{2.5cm}{4.5cm}{0.5cm}{2.5cm}{Lightning-42-F1000.pdf}

\session{Software Interoperability}{Day Two, 12 July 2014, 10:45 -- 12:30}{Session chair: Raoul Bonnal}

\embed{2.5cm}{3cm}{1.5cm}{2.5cm}{Interoperability-04-Pathview.pdf}
\embed{2.5cm}{2.5cm}{1cm}{2.5cm}{Interoperability-13-Mobyle2.pdf}
\embed{2.5cm}{2.2cm}{1cm}{3cm}{Interoperability-34-OWL-computing.pdf}
\embed{2.5cm}{2.5cm}{0.5cm}{3.5cm}{Interoperability-14-Taverna-player.pdf}
\embed{2.5cm}{2.5cm}{1.5cm}{3cm}{Interoperability-40-Small-Tools.pdf}

\session{Open Science and Reproducible Research}{Day Two, 12 July 2014, 14:00 -- 15:30}{Session chair: Hilmar Lapp}

\embed{2.5cm}{10cm}{2cm}{2.5cm}{OpenScience-11-SEEK.pdf}
\embed{2.5cm}{5cm}{2.5cm}{2.5cm}{OpenScience-31-Arvados.pdf}
\embed{3cm}{2.5cm}{1cm}{2.5cm}{OpenScience-35-Galaxy-community.pdf}
\embed{2.5cm}{5cm}{2.5cm}{2.5cm}{OpenScience-23-Community-Pipelines.pdf}
\embed{2.5cm}{3.25cm}{2.5cm}{3cm}{OpenScience-26-Open-Tree-of-Life.pdf}

\end{document}
